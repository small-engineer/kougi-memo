\documentclass{beamer}
\usetheme{metropolis} % Metropolisテーマの使用
\usepackage{amsmath}  % 数式サポート
\usepackage{graphicx} % 画像サポート
\usepackage{hyperref} % ハイパーリンク
\usepackage{tikz}     % 図形サポート
\usepackage{xcolor}   % 色設定

% テーマカラー設定
\definecolor{turquoise}{HTML}{40e0d0}
\definecolor{tomato}{HTML}{ff6347}
\setbeamercolor{alerted text}{fg=tomato}
\setbeamercolor{frametitle}{bg=turquoise}
\begin{document}
\setmainfont{HackGen}


\title{販売計画と戦略}
\author{あなたの名前}
\date{\today}

\begin{document}

% タイトルページを生成
\maketitle

\section{STP}
\begin{itemize}
    \item \textbf{販売ターゲットは誰か?}
    \item \textbf{製品にどのような優位性があるか?}
\end{itemize}

\section{4P4C}
\subsection{Product(Value)}
\begin{itemize}
    \item 名称、ネーミング
    \item キャッチコピー
    \item 製品ジャンル
    \item その製品の特長、購入者にとって何がベネフィットか。
\end{itemize}

\subsection{Price(Cost)}
\begin{itemize}
    \item 販売予定価格
\end{itemize}

\subsection{Place(Convenience)}
\begin{itemize}
    \item 販売サイト
    \item SUZURI等
    \item その他の販売場所、販売方法
\end{itemize}

\subsection{Promotion(Communication)}
\begin{itemize}
    \item 宣伝方法
\end{itemize}

\section{販売計画}
\begin{itemize}
    \item 週単位
    \item 月単位
    \item 最終目標
\end{itemize}
\begin{flushright}
(10/28の週を第1週として考える)
\end{flushright}

\end{document}